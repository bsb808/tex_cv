\documentclass[11pt]{article}

% Created by @btskinner (GitHub)

% -- NOTES -----------------------------

% Adjust this template as needed to create a CV that meets your
% specific needs. Add / delete / move the sections in the
% document body below to make a personalized CV. In general, sections
% that need user input versus those that should be left as is (except
% by more advanced users who want to fiddle with details of the
% template) are separated like this:
%
% - Commented sections using "====" are for user modification.
% - Commented sections using "----" do not need to changed.
%
% This template has been munged together from lots of little code
% snippets over the years, so I can't take complete credit. I've tried
% to attribute blocks of code I took from elsewhere.

% ==============================================================================
% USER SETTINGS: CHANGE THINGS HERE
% ==============================================================================

% ======================================
% PERSONAL INFO
% ======================================

% Change the value in the second set of {} to you information. If you
% want multiple lines (like for the address), use \\ to split lines.

\newcommand{\myname}{Brian S. Bingham}     % your name
%\newcommand{\websiteurl}{{https://github.com/btskinner/tex_cv}} % your site url
\newcommand{\githuburl}{https://github.com/bsb808}
\newcommand{\linkedinurl}{https://www.linkedin.com/in/briansbingham/}
\newcommand{\websitename}{me.com}       % short url name for printing
\newcommand{\phone}{(831) 760-1670}       % your phone number
\newcommand{\email}{briansbingham@gmail.com}  % your email
\newcommand{\address}{                  % NB: Assumes 3 lines
  Watkins Hall 331\\                    % {BUILDING} \\
  1 University Way\\                       % {ADDRESS} \\
  Monterey, CA 93950                        % {CITY, ST ZIP}
}

% ======================================
% FILTER TO ITEMS AFTER THIS YEAR
% ======================================

% If you want to list publications / work products only from the past
% XX years, set this value to the 4-digit year that's one less than
% the first year you want.
%
% If only 2015 - present ---> set to 2014

\newcommand{\recentyear}{1900}  % artificially low year to include everything

% ------------------------------------------------------------------------------
% HEADER STUFF (CAN LEAVE ALONE UNLESS YOU WANT TO PERSONALIZE)
% ------------------------------------------------------------------------------

% -------------------------------
% Packages
% -------------------------------

\usepackage[margin=1in]{geometry} % for margins
\usepackage[american]{babel}      % for language
\usepackage{csquotes}             % idk, babel/biber like it
\usepackage[style=apa,
backend=biber,
sortcites=true,
sorting=ydmdnt,
language=american]{biblatex}      % for reference sections
\usepackage{hyperref}             % to add links
\usepackage{fancyhdr}             % for better headers/footers
\usepackage{titlesec}             % to adjust section headers
\usepackage{tabularx}             % for fluid tables
\usepackage{datetime}             % to make better date formate
\usepackage[inline]{enumitem}     % more control over lists

% -------------------------------
% Macros
% -------------------------------

\newcommand{\RR}{\raggedright\arraybackslash} % left justified
\newcommand{\RL}{\raggedleft\arraybackslash}  % right justified

% for tables to keep consistent alignment across sections
\newcolumntype{\twocols}{>{\RR}p{1.25in}>{\RR}X}
\newcolumntype{\threecols}{>{\RR}p{1.25in}>{\RR}X>{\RL}p{1in}}

% -------------------------------
% BibLaTeX sorting scheme
% -------------------------------

% Create a new sorting scheme that will sort in reverse chronological
% order, taking months into account correctly (i.e., in chronological
% order and not alphabetical order)

\DeclareSortingScheme{ydmdnt}{
  \sort{
    \field{presort}
  }
  \sort[final]{
    \field{sortkey}
  }
  \sort[direction=descending]{
    \field{sortyear}
    \field{year}
    \literal{9999}
  }
  \sort[direction=descending]{
    \field[padside=left,padwidth=2,padchar=0]{month}
    \literal{99}
  }
  \sort[direction=descending]{
    \field[padside=left,padwidth=2,padchar=0]{day}
    \literal{99}
  }
  \sort{
    \field{sortname}
    \field{author}
    \field{editor}
    \field{translator}
    \field{sorttitle}
    \field{title}
  }
  \sort{
    \field{sorttitle}
  }
  \sort[direction=descending]{
    \field[padside=left,padwidth=4,padchar=0]{volume}
    \literal{9999}
  }
}

% -------------------------------
% BibLaTeX options
% -------------------------------

% language mapping
\DeclareLanguageMapping{american}{american-apa}

% ignore addendum and note fields that are sometimes useful, but not here
\DeclareSourcemap{
  \maps[datatype=bibtex]{
    \map{
      \step[fieldset=addendum, null]
      \step[fieldset=note, null]
    } 
  }
}

% macro so titles are converted to doi, url, or isbn link (if available)
% h\t https://tex.stackexchange.com/a/48409
\ExecuteBibliographyOptions{doi=false,url=false,isbn=false}
\newbibmacro{string+doiurlisbn}[1]{%
  \iffieldundef{doi}{%
    \iffieldundef{url}{%
      \iffieldundef{isbn}{%
        \iffieldundef{issn}{%
          #1%
        }{%
          \href{http://books.google.com/books?vid=ISSN\thefield{issn}}{#1}%
        }%
      }{%
        \href{http://books.google.com/books?vid=ISBN\thefield{isbn}}{#1}%
      }%
    }{%
      \href{\thefield{url}}{#1}%
    }%
  }{%
    \href{http://dx.doi.org/\thefield{doi}}{#1}%
  }%
}

\DeclareFieldFormat{title}{\usebibmacro{string+doiurlisbn}{\mkbibemph{#1}}}
\DeclareFieldFormat[article,incollection]{title}%
{\usebibmacro{string+doiurlisbn}{\mkbibquote{#1}}}

% no letter for same year entries
\defbibenvironment{mybib}
{\list
  {}
  {\setlength{\leftmargin}{\bibhang}%
    \setlength{\itemindent}{-\leftmargin}%
    \setlength{\itemsep}{\bibitemsep}%
    \setlength{\parsep}{\bibparsep}}}
{\endlist}
{\clearfield{extradate}\item}

% filter by year (only print those after year value in \recentyear
% macro above)
\defbibcheck{recent}{
  \iffieldint{year}
  {\ifnumless{\thefield{year}}{\recentyear}
    {\skipentry}
    {}}
  {\skipentry}
}

% -------------------------------
% Document options
% -------------------------------

% keep lists tight (no extra whitespace)
\setlist{nolistsep}

% This section determines how the sections are displayed. Adjust these
% as you like using the titlesec package options.

% section
\titleformat{\section}[display]
{\Large\bfseries}
{\filleft\Large\sectiontitlename~\thesection}
{}
{\vspace{-1ex}}
[]

% subsections
\titleformat{\subsection}[display]
{\bfseries}
{\filleft\subsectiontitlename~\thesubsection}
{}
{}
[\vspace{-1ex}]

% subsubsections
\titleformat{\subsubsection}[display]
{}
{\filleft\subsubsectiontitlename~\thesubsubsection}
{}
{}
[\vspace{-1ex}]

% -------------------------------
% Headers and Footers
% -------------------------------

% All pages after the first will have your name and page number in the
% upper right corner; all pages will have the date of compilation (the
% latest update) in the bottom right corner. You can move these.

\fancypagestyle{first}{
   \fancyhead[R]{}
   \fancyhead[L]{}
   \fancyfoot[R]{{\itshape Updated: \today}}
   \fancyfoot[L]{}
   \fancyfoot[C]{}
   \renewcommand{\headrulewidth}{0.0pt}
   \renewcommand{\footrulewidth}{0.0pt}}

\fancypagestyle{rest}{
   \fancyhead[R]{{\itshape \myname\, $\mid$ \thepage}}
   \fancyhead[L]{}
   \fancyfoot[R]{{\itshape Updated: \today}}
   \fancyfoot[L]{}
   \fancyfoot[C]{}
   \renewcommand{\headrulewidth}{0.0pt}
   \renewcommand{\footrulewidth}{0.0pt}}

%%%%%%%%%%%%%%%%%%%%%%%%%%%%%%%%%%%%%%%%%%%%%%%%%%%%%%%%%%%%%%%%%%%%%%%%%%%%%%%%
% BEGIN DOCUMENT
%%%%%%%%%%%%%%%%%%%%%%%%%%%%%%%%%%%%%%%%%%%%%%%%%%%%%%%%%%%%%%%%%%%%%%%%%%%%%%%%

% ======================================
% Bibliography
% ======================================

% path to your CV bibliography (.bib) file; should be same name as
% your *.tex file if you want to use the makefile
\addbibresource{./cv.bib}
 
\pagestyle{rest}
\begin{document}
\thispagestyle{first}

% -------------------------------
% Front page header
% -------------------------------

% Adjust this as necessary if you need to add or delete items from the
% top. You can also create your own. The structure right now is:
%
% |----------------------------------------------|
% |       Centered box across page for Name      |
% |----------------------------------------------|
% | Left-align 1/2 box    | Right-align 1/2 box  |
% |                       |                      |
% |  * building           |             phone *  |
% |  * street             |           website *  |
% |  * city, ST ZIP       |             email *  |
% |----------------------------------------------|
% _____________________ HRULE ___________________

\hspace*{-\parindent}%
\begin{center}
  \vspace{-2em}
  {\LARGE\scshape \myname} \\
\end{center}
\begin{minipage}[t]{0.4\linewidth}
\address
\end{minipage}
\hspace*{-\parindent}%
\begin{minipage}[t]{0.6\linewidth}
\begin{flushright}
  \textit{Email}: \href{mailto:\email}{\email} \\
  \textit{Phone}: \phone \\
  \textit{GitHub}: \href{\githuburl}{\githuburl} \\
  \textit{LinkedIn}: \href{\linkedinurl}{\linkedinurl} \\
  %\textit{tel}: \phone \\
  %\textit{web}: \href{\websiteurl}{\websitename} \\
  %\textit{email}: \href{mailto:\email}{\email}
\end{flushright}
\end{minipage}
\begin{center}
  \vspace{-.5em}
  \rule{\textwidth}{1pt}  
\end{center}

% ==============================================================================
% CV BODY: CHANGE THINGS STARTING HERE
% ==============================================================================

% ======================================
% EDUCATION
% ======================================

% Use tabularx (for flexible spacing) to align nicely

\section*{Education}
\renewcommand{\arraystretch}{1.5} % Default value: 1
\begin{tabularx}{\linewidth}{>{\RR}p{0.5in}>{\RR}X>{\RL}p{1in}}
  2003 & \renewcommand{\arraystretch}{1} \begin{tabular}[t]{@{}l@{}}Massachusetts Institute Technology, Mechanical Engineering.
          \\ Committee: Warren Seering, David Mindell and Dana Yoerger.
          \\ Dissertation: \textit{Precision Autonomous Underwater Navigation}.
          \end{tabular} & Ph.D. \\ % or YYYY (expected)
          \renewcommand{\arraystretch}{1.5} 
  1998 & \renewcommand{\arraystretch}{1}  \begin{tabular}[t]{@{}l@{}}
            Massachusetts Institute Technology, Mechanical Engineering.
            \\ Thesis: \textit{Structural-acoustic design and control of } 
            \\ \hspace{4em}\textit{an integrally actuated composite panel.}
          \end{tabular} & M.S. \\
          \renewcommand{\arraystretch}{1.5} 
  1996 & \renewcommand{\arraystretch}{1} \begin{tabular}[t]{@{}l@{}}Missouri University of Science and Technology, Mechanical Engineering. \end{tabular} & B.S. \\
\end{tabularx}

% ======================================
% RESEARCH INTERESTS
% ======================================

% This will make a horizontal list of items separated by semi-colons. 

\section*{Research Interests}

\hspace{-.25em}\begin{itemize*}[itemjoin={{; }}, label={}]
\item Robotics and autonomous vehicles
\item Navigation and control
\item Reinforcement learning
\end{itemize*}

% ======================================
% APPOINTMENTS
% ======================================

% Uncomment to add if you have a current appointment.

\section*{Academic Appointments}
\begin{tabularx}{\linewidth}{\twocols}
  2019--present & Professor, Department of Mechanical and Aerospace Engineering, Naval Postgraduate School. \\
  2022--2024& Department Chair, Mechanical and Aerospace Engineering, Naval Postgraduate School. \\
  2015--2019 & Associate Professor, Department of Mechanical and Aerospace Engineering, Naval Postgraduate School. \\
  2013--2015 & Associate Professor, Department of Mechanical Engineering, University of Hawaii at Manoa. \\
  2009--2013 & Assistant Professor, Department of Mechanical Engineering, University of Hawaii at Manoa. \\
  2005--2008 & Assistant Professor, Franklin W. Olin College of Engineering \\
  2004--2005 & Postdoctoral Investigator, Woods Hole Oceanographic Institution. 
 \end{tabularx}


% ======================================
% PUBLICATIONS
% ======================================

% Use the same keyword that you've placed in your bib file entry to
% filter only those references you want to place here.

\begin{refsection}
\section*{Publications}
\nocite{*}                          % cite everything
\printbibliography[heading = none,  % no heading (e.g., "References")
keyword = article,                  % FILTER BY THIS KEYWORD
check = recent,                     % check year filter from above
env = mybib]                        % use mybib style
\end{refsection}

% ======================================
% SUBMITTED PAPERS
% ======================================

\begin{refsection}
\section*{Submitted Papers}
\nocite{*}
\printbibliography[heading = none,
keyword = submitted,
check = recent,
env = mybib]
\end{refsection}

% ======================================
% WORKING PAPERS
% ======================================

\begin{refsection}
\section*{Working Papers}
\nocite{*}
\printbibliography[heading = none,
keyword = working,
check = recent,
env = mybib]
\end{refsection}

% ======================================
% PRESENTATIONS
% ======================================

% This section differentiates between conference presentations and
% invited presentations. If you only have one or the other, just
% delete the subsections and have one PUBLICATIONS section.

\section*{Presentations}
\subsection*{Peer-reviewed conference presentations}
\begin{refsection}
\nocite{*}
\printbibliography[heading = none,
keyword = conference,               % must have "conference" AND 
keyword = peer,                     % "peer" to differentiate from "invited"
env = mybib,
check = recent]                     
\end{refsection}                    

\subsection*{Invited presentations}
\begin{refsection}
\nocite{*}
\printbibliography[heading = none,
keyword = conference,               % must have "conference" AND 
keyword = invited,                  % "invited" to differentiate from "peer"
env = mybib,
check = recent]                   
\end{refsection}                   

% ======================================
% TEACHING EXPERIENCE
% ======================================

% Add what makes sense for your field, but if you have links to course
% materials online, be sure to incorporate them below. (NB: the
% backslash on the underscore just b/c TeX treats that differently)

\section*{Teaching}

\subsection*{Instructor}
\subsubsection*{University $\mid$ Department}
\begin{tabularx}{\linewidth}{\twocols}
YYYY&COURSE NUM: \textit{Course name} \\
&\hspace{1em}One sentence description (optional)  \\
&\hspace{1em}{\itshape Website/Course materials:}
\href{https://github.com/btskinner/tex\_cv}{github.com/btskinner/tex\_cv} \\
\end{tabularx}

\subsection*{Teaching Assistant}
\subsubsection*{University $\mid$ Department}
\begin{tabularx}{\linewidth}{\twocols}
YYYY & COURSE NUM: \textit{Course name} \\
&\hspace{1em}	\textit{Instructor:} Instructor of record  \\
\end{tabularx}

% ======================================
% FELLOWSHIPS AND HONORS
% ======================================

% Again, use a table for nicely aliged elements.

\section*{Awards, Fellowships, and Honors}
\begin{tabularx}{\linewidth}{\threecols}
YYYY & Honor \\
YYYY - YYYY & Fellowship & \$ AMOUNT \\ % if normal in your field

\end{tabularx}

% ======================================
% PROFESSIONAL MEMBERSHIPS
% ======================================

% This time the list is separated by bullets as an example; use
% whatever seems most aesthetically pleasing to your eyes.

\section*{Professional Memberships}

\hspace{-.25em}\begin{itemize*}[itemjoin={{ $\bullet$}}, label={}]
\item Organization 1
\item Organization 2
\item Organization 3
\item Organization 4
\item Organization 5
\end{itemize*}

% ======================================
% OTHER
% ======================================

% Add other sections as needed...

%%%%%%%%%%%%%%%%%%%%%%%%%%%%%%%%%%%%%%%%%%%%%%%%%%%%%%%%%%%%%%%%%%%%%%%%%%%%%%%%
% END DOCUMENT
%%%%%%%%%%%%%%%%%%%%%%%%%%%%%%%%%%%%%%%%%%%%%%%%%%%%%%%%%%%%%%%%%%%%%%%%%%%%%%%%

\end{document}
